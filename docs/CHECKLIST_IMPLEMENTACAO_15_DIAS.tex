% ============================================================
% NEOVIEW - CHECKLIST DE IMPLEMENTAÇÃO BACKEND
% ============================================================
% Período: 15 dias úteis
% Estrutura: Frontend preparado + Integração Backend via Copilot
% ============================================================

\documentclass[12pt,a4paper]{article}
\usepackage[utf8]{inputenc}
\usepackage[portuguese]{babel}
\usepackage{geometry}
\usepackage{enumitem}
\usepackage{xcolor}
\usepackage{tikz}
\usepackage{hyperref}
\usepackage{fancyhdr}
\usepackage{tcolorbox}

\geometry{margin=2cm}
\hypersetup{colorlinks=true,linkcolor=blue,urlcolor=blue}

% Cores
\definecolor{neogreen}{RGB}{0, 128, 64}
\definecolor{neoblue}{RGB}{0, 64, 128}
\definecolor{neoorange}{RGB}{255, 128, 0}

% Checkbox commands
\newcommand{\checkbox}{$\square$}
\newcommand{\checkedbox}{$\boxtimes$}

\title{\textbf{NeoView - Checklist de Implementação Backend}\\
\large Período: 15 Dias Úteis}
\author{Projeto: Plataforma de Relatórios Corporativos Neoenergia}
\date{\today}

\begin{document}

\maketitle

\begin{tcolorbox}[colback=blue!5!white,colframe=neoblue,title=\textbf{Visão Geral}]
Este documento apresenta o checklist completo para implementação do backend do NeoView, considerando que o frontend já está preparado e será integrado via GitHub + Copilot. A implementação utiliza PostgreSQL (Lovable Cloud/Supabase) com possibilidade de sincronização com SAP HANA via DBeaver.
\end{tcolorbox}

% ============================================================
\section{Estrutura Atual do Frontend}
% ============================================================

O frontend foi preparado com os seguintes componentes prontos para integração:

\subsection{Tipos TypeScript (src/types/backend.ts)}
\begin{itemize}[label=\checkbox]
    \item \texttt{User, UserRole, UserRoleAssignment} - Autenticação e roles
    \item \texttt{CompanyEntity, SuperintendenceEntity, ManagementEntity, ProjectEntity} - Hierarquia
    \item \texttt{IndicatorEntity} - Indicadores e métricas
    \item \texttt{ReportEntity, ReportVersion, ReportApproval} - Relatórios e aprovações
    \item \texttt{ChatMessage, ChatSession, SearchSource} - Chatbot
    \item \texttt{AuditLog, SystemSetting, UserPreference} - Configurações
\end{itemize}

\subsection{Hooks de Integração}
\begin{itemize}[label=\checkbox]
    \item \texttt{useAuth.tsx} - Autenticação completa com mock
    \item \texttt{useReports.ts} - CRUD de relatórios
    \item \texttt{useApprovals.ts} - Sistema de aprovação
    \item \texttt{useChatbot.ts} - Busca semântica com IA
\end{itemize}

\subsection{Páginas Funcionais}
\begin{itemize}[label=\checkbox]
    \item \texttt{/dashboard} - Dashboard principal com navegação hierárquica
    \item \texttt{/companies} - Gerenciamento de empresas
    \item \texttt{/reports} - Listagem e upload de relatórios
    \item \texttt{/indicators} - Visualização de indicadores
    \item \texttt{/approvals} - Aprovação de relatórios (supervisores)
    \item \texttt{/settings} - Configurações e supervisores
\end{itemize}

% ============================================================
\section{Cronograma de Implementação - 15 Dias}
% ============================================================

\subsection{Dias 1-2: Configuração Inicial do Banco}

\begin{tcolorbox}[colback=green!5!white,colframe=neogreen]
\textbf{Objetivo:} Criar estrutura base do PostgreSQL
\end{tcolorbox}

\begin{itemize}[label=\checkbox]
    \item Criar tabela \texttt{profiles} vinculada a \texttt{auth.users}
    \item Criar ENUM \texttt{app\_role} com valores: admin, supervisor, analyst, viewer
    \item Criar tabela \texttt{user\_roles} com RLS
    \item Criar função \texttt{has\_role()} SECURITY DEFINER
    \item Configurar trigger para criar profile automaticamente no signup
    \item Testar autenticação básica
\end{itemize}

\textbf{SQL de Referência:}
\begin{verbatim}
CREATE TYPE public.app_role AS ENUM ('admin', 'supervisor', 'analyst', 'viewer');

CREATE TABLE public.profiles (
    id UUID PRIMARY KEY DEFAULT gen_random_uuid(),
    user_id UUID REFERENCES auth.users(id) ON DELETE CASCADE UNIQUE,
    full_name TEXT,
    department TEXT,
    phone TEXT,
    avatar_url TEXT,
    created_at TIMESTAMPTZ DEFAULT now(),
    updated_at TIMESTAMPTZ DEFAULT now()
);

CREATE TABLE public.user_roles (
    id UUID PRIMARY KEY DEFAULT gen_random_uuid(),
    user_id UUID REFERENCES public.profiles(user_id) ON DELETE CASCADE,
    role app_role NOT NULL,
    company_id UUID,
    UNIQUE(user_id, role, company_id)
);
\end{verbatim}

% ============================================================
\subsection{Dias 3-4: Hierarquia Organizacional}

\begin{tcolorbox}[colback=green!5!white,colframe=neogreen]
\textbf{Objetivo:} Criar tabelas de hierarquia (Company → Superintendence → Management → Project)
\end{tcolorbox}

\begin{itemize}[label=\checkbox]
    \item Criar tabela \texttt{companies}
    \item Criar tabela \texttt{superintendences} com FK para companies
    \item Criar tabela \texttt{managements} com FK para superintendences
    \item Criar tabela \texttt{projects} com FK para managements
    \item Aplicar RLS em todas as tabelas (SELECT para authenticated)
    \item Popular com dados iniciais das empresas Neoenergia
    \item Atualizar hook \texttt{useAuth} para buscar dados reais
\end{itemize}

% ============================================================
\subsection{Dias 5-6: Indicadores}

\begin{tcolorbox}[colback=green!5!white,colframe=neogreen]
\textbf{Objetivo:} Sistema de indicadores com histórico
\end{tcolorbox}

\begin{itemize}[label=\checkbox]
    \item Criar tabela \texttt{indicators} com FK para projects
    \item Criar tabela \texttt{indicator\_history} para histórico de valores
    \item Criar ENUMs: \texttt{indicator\_type}, \texttt{indicator\_trend}
    \item Criar trigger para registrar histórico em updates
    \item Implementar RLS: SELECT para todos, UPDATE para supervisors
    \item Popular com indicadores exemplo (DEC, FEC, Perdas, etc.)
    \item Conectar página \texttt{/indicators} aos dados reais
\end{itemize}

% ============================================================
\subsection{Dias 7-9: Sistema de Relatórios PDF}

\begin{tcolorbox}[colback=green!5!white,colframe=neogreen]
\textbf{Objetivo:} Upload, versionamento e aprovação de PDFs
\end{tcolorbox}

\begin{itemize}[label=\checkbox]
    \item Criar bucket \texttt{reports} no Storage
    \item Configurar políticas de acesso ao bucket
    \item Criar tabela \texttt{reports} com metadados
    \item Criar tabela \texttt{report\_versions} para versionamento
    \item Criar ENUM \texttt{report\_status}: draft, pending\_approval, approved, rejected, archived
    \item Implementar upload via Edge Function ou direto do frontend
    \item Criar tabela \texttt{report\_approvals}
    \item Atualizar \texttt{useReports} para calls reais
\end{itemize}

\textbf{Fluxo de Upload:}
\begin{enumerate}
    \item Frontend captura arquivo PDF
    \item Upload para \texttt{storage.from('reports')}
    \item Inserir registro em \texttt{reports} com status \texttt{draft}
    \item Usuário submete para aprovação → status \texttt{pending\_approval}
    \item Supervisor aprova/rejeita → atualiza status
\end{enumerate}

% ============================================================
\subsection{Dias 10-11: Sistema de Aprovação}

\begin{tcolorbox}[colback=green!5!white,colframe=neogreen]
\textbf{Objetivo:} Workflow de aprovação por supervisores
\end{tcolorbox}

\begin{itemize}[label=\checkbox]
    \item Criar tabela \texttt{area\_supervisors} (quem pode aprovar onde)
    \item Implementar função \texttt{is\_supervisor\_for(entity\_type, entity\_id)}
    \item Criar políticas RLS baseadas em área
    \item Implementar notificação de nova aprovação pendente
    \item Criar trigger para atualizar status do relatório após decisão
    \item Atualizar \texttt{useApprovals} para dados reais
    \item Testar fluxo completo: upload → submissão → aprovação
\end{itemize}

% ============================================================
\subsection{Dias 12-13: Chatbot com Busca Semântica}

\begin{tcolorbox}[colback=green!5!white,colframe=neogreen]
\textbf{Objetivo:} Integrar IA para busca em linguagem natural
\end{tcolorbox}

\begin{itemize}[label=\checkbox]
    \item Criar Edge Function \texttt{/functions/chat}
    \item Configurar chamada ao Lovable AI Gateway
    \item Implementar busca de contexto (indicadores, relatórios)
    \item Criar tabelas \texttt{chat\_sessions} e \texttt{chat\_messages}
    \item Implementar prompt de sistema com instruções
    \item Atualizar \texttt{useChatbot} para Edge Function real
    \item Testar buscas: "Qual o DEC da Coelba?", "Relatórios de perdas"
\end{itemize}

\textbf{Estrutura da Edge Function:}
\begin{verbatim}
// supabase/functions/chat/index.ts
const response = await fetch('https://ai.gateway.lovable.dev/v1/chat/completions', {
  method: 'POST',
  headers: {
    'Authorization': `Bearer ${Deno.env.get('LOVABLE_API_KEY')}`,
    'Content-Type': 'application/json',
  },
  body: JSON.stringify({
    model: 'google/gemini-3-flash-preview',
    messages: [...history, { role: 'user', content: userMessage }],
    max_tokens: 1000,
  }),
});
\end{verbatim}

% ============================================================
\subsection{Dias 14-15: Auditoria, Testes e Deploy}

\begin{tcolorbox}[colback=green!5!white,colframe=neogreen]
\textbf{Objetivo:} Finalização, logs e qualidade
\end{tcolorbox}

\begin{itemize}[label=\checkbox]
    \item Criar tabela \texttt{audit\_logs}
    \item Implementar trigger para log automático de operações
    \item Criar tabelas \texttt{system\_settings} e \texttt{user\_preferences}
    \item Revisar todas as políticas RLS
    \item Executar linter de segurança do Supabase
    \item Testar fluxos end-to-end
    \item Documentar endpoints e estrutura
    \item Publicar versão de produção
\end{itemize}

% ============================================================
\section{Integração SAP HANA via DBeaver}
% ============================================================

\begin{tcolorbox}[colback=orange!5!white,colframe=neoorange,title=\textbf{Sincronização PostgreSQL → SAP HANA}]
Esta seção detalha como materializar os dados no SAP HANA para uso em sistemas corporativos.
\end{tcolorbox}

\subsection{Mapeamento de Tipos}

\begin{center}
\begin{tabular}{|l|l|l|}
\hline
\textbf{PostgreSQL} & \textbf{SAP HANA} & \textbf{Notas} \\
\hline
UUID & NVARCHAR(36) & Converter para string \\
TIMESTAMPTZ & TIMESTAMP & Timezone UTC \\
TEXT & NCLOB & Para textos longos \\
JSONB & NCLOB & Serializar como JSON string \\
BOOLEAN & TINYINT & 0/1 \\
NUMERIC & DECIMAL(18,4) & Precisão adequada \\
\hline
\end{tabular}
\end{center}

\subsection{Schema SAP HANA (NEOVIEW)}

\begin{itemize}[label=\checkbox]
    \item Criar schema \texttt{NEOVIEW}
    \item Criar tabela \texttt{NEOVIEW.T\_USERS}
    \item Criar tabela \texttt{NEOVIEW.T\_COMPANIES}
    \item Criar tabela \texttt{NEOVIEW.T\_HIERARCHY} (desnormalizada)
    \item Criar tabela \texttt{NEOVIEW.T\_INDICATORS}
    \item Criar tabela \texttt{NEOVIEW.T\_REPORTS}
    \item Criar tabela \texttt{NEOVIEW.T\_SYNC\_LOG}
    \item Criar views analíticas
\end{itemize}

\subsection{Estratégias de Sincronização}

\textbf{1. Carga Inicial (Full Load):}
\begin{itemize}[label=\checkbox]
    \item Exportar dados via \texttt{pg\_dump} ou queries
    \item Transformar UUIDs e JSONBs
    \item Importar via DBeaver ou script ETL
\end{itemize}

\textbf{2. Carga Incremental (Delta - a cada 15min):}
\begin{itemize}[label=\checkbox]
    \item Criar Edge Function \texttt{/functions/sync-hana}
    \item Rastrear \texttt{updated\_at > last\_sync}
    \item Gerar payload JSON para consumo
    \item Registrar em \texttt{NEOVIEW.T\_SYNC\_LOG}
\end{itemize}

\textbf{3. Tempo Real (CDC - opcional):}
\begin{itemize}[label=\checkbox]
    \item Configurar Realtime no Supabase para tabelas críticas
    \item Criar webhook para enviar mudanças
    \item Processar via middleware ou API intermediária
\end{itemize}

% ============================================================
\section{Arquivos de Referência no Projeto}
% ============================================================

\begin{tcolorbox}[colback=blue!5!white,colframe=neoblue]
\textbf{Ao clonar do GitHub, consulte estes arquivos:}
\end{tcolorbox}

\begin{itemize}
    \item \texttt{src/types/backend.ts} - Todos os tipos com comentários de mapeamento SQL
    \item \texttt{src/hooks/useAuth.tsx} - Instruções de integração nos comentários
    \item \texttt{src/hooks/useReports.ts} - Exemplos de queries Supabase
    \item \texttt{src/hooks/useApprovals.ts} - Lógica de aprovação
    \item \texttt{src/hooks/useChatbot.ts} - Estrutura da Edge Function de chat
\end{itemize}

% ============================================================
\section{Comandos Úteis para Copilot}
% ============================================================

\begin{tcolorbox}[colback=gray!10!white,colframe=gray!50!black]
\textbf{Prompts recomendados para o GitHub Copilot:}
\end{tcolorbox}

\begin{enumerate}
    \item "Substituir mock data em useAuth por chamadas reais ao Supabase"
    \item "Criar migration SQL para tabela reports baseada em ReportEntity"
    \item "Implementar upload de PDF para Supabase Storage"
    \item "Criar Edge Function para chat com Lovable AI Gateway"
    \item "Adicionar RLS policies para tabela reports permitindo SELECT para authenticated e UPDATE para owner"
\end{enumerate}

% ============================================================
\section{Checklist Final de Qualidade}
% ============================================================

\begin{itemize}[label=\checkbox]
    \item Todas as tabelas têm RLS habilitado
    \item Roles são verificados via função server-side (não client-side)
    \item Secrets estão em variáveis de ambiente
    \item Edge Functions validam autenticação
    \item Uploads de PDF limitados a mime-type application/pdf
    \item Chatbot tem limite de tokens configurado
    \item Logs de auditoria funcionando
    \item Testes de fluxo end-to-end passando
    \item Documentação de API atualizada
\end{itemize}

\vspace{1cm}
\begin{center}
\textit{Documento gerado automaticamente pelo NeoView}\\
\textit{Versão 1.0 - \today}
\end{center}

\end{document}
